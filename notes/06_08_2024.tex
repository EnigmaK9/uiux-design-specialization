\documentclass{article}
\usepackage[utf8]{inputenc}

\title{Proceso para fundamentar una app móvil}
\author{Carlos Padilla}
\date{2024-06-08}

\begin{document}

\maketitle

\section{Imagen para reconocer películas}
En esta sección se presentaría una imagen donde los usuarios deben reconocer diferentes películas. 

\section{UX Writing}
El UX Writing se refiere a la redacción de los textos que guían a los usuarios a través de la interfaz de una aplicación o sitio web. Es crucial para crear una experiencia de usuario coherente y efectiva.

\section{Elementos de UX según JJ Garrett}
De acuerdo con JJ Garrett, las etapas de un desarrollo deben trabajarse por capas para identificar cada elemento.

\subsection{Capas de Garrett}

\subsubsection{Superficie}
Esta capa se refiere al diseño visual de la interfaz, incluyendo colores, tipografía, y elementos gráficos.

\subsubsection{Esqueleto}
La ubicación de los diversos elementos que tienen las páginas y la relación entre ellos. Ejemplo: botones, texto, etc.

\subsubsection{Estructura}
Relación entre las páginas del sitio, los flujos entre sí, estructura de navegación, etc.

\subsubsection{Alcance}
Funciones y características del sitio. Es el "qué" del proyecto.

\subsubsection{Estrategia}
Consiste en aclarar qué esperan del sitio los dueños, gestores, los usuarios. Es el "por qué".

\section{Estrategia}
\subsection{Necesidades del Usuario (Software / Desarrollo)}
\begin{itemize}
    \item Investigación de usuarios
    \item Pruebas de usabilidad: entrevistas, focus groups, etc.
\end{itemize}

\subsection{Objetivos del Sitio (Negocio) (Software / Desarrollo)}
\begin{itemize}
    \item Metas financieras
    \item Limitantes tecnológicas
    \item Objetivos de las personas
\end{itemize}

\end{document}
