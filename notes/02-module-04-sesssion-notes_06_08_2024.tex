\documentclass{article}
\usepackage[utf8]{inputenc}

\title{Proceso para fundamentar una app móvil}
\author{Carlos Padilla}
\date{2024-06-08}

\begin{document}

\maketitle

\section{Imagen para reconocer películas}
En esta sección se presentaría una imagen donde los usuarios deben reconocer diferentes películas. 

\section{UX Writing}
El UX Writing se refiere a la redacción de los textos que guían a los usuarios a través de la interfaz de una aplicación o sitio web. Es crucial para crear una experiencia de usuario coherente y efectiva.

\section{Elementos de UX según JJ Garrett}
De acuerdo con JJ Garrett, las etapas de un desarrollo deben trabajarse por capas para identificar cada elemento.

\subsection{Capas de Garrett}

\subsubsection{Superficie}
Esta capa se refiere al diseño visual de la interfaz, incluyendo colores, tipografía, y elementos gráficos.

\subsubsection{Esqueleto}
La ubicación de los diversos elementos que tienen las páginas y la relación entre ellos. Ejemplo: botones, texto, etc.

\subsubsection{Estructura}
Relación entre las páginas del sitio, los flujos entre sí, estructura de navegación, etc.

\subsubsection{Alcance}
Funciones y características del sitio. Es el "qué" del proyecto.

\subsubsection{Estrategia}
Consiste en aclarar qué esperan del sitio los dueños, gestores, los usuarios. Es el "por qué".

\section{Estrategia}

\subsection{Necesidades del Usuario (Software / Desarrollo)}
\begin{itemize}
    \item Investigación de usuarios
    \item Pruebas de usabilidad: entrevistas, focus groups, etc.
\end{itemize}

\subsection{Objetivos del Sitio (Negocio) (Software / Desarrollo)}
\begin{itemize}
    \item Metas financieras
    \item Limitantes tecnológicas
    \item Objetivos de las personas
\end{itemize}

\subsection{Preguntas que funcionan}
\begin{itemize}
    \item ¿Qué debe lograr la app para el negocio?
    \item ¿Quién eres?
\end{itemize}

\subsection{Preguntas que debemos hacernos}
\begin{itemize}
    \item Se debe contar con una visión alineada del producto o app para tener:
    \item Especificaciones y requerimientos
    \item Entendimiento general de cuáles serán las funciones, el contenido y quién lo generará
\end{itemize}

\subsection{Cuál es el valor que provee}
\begin{itemize}
    \item Especificar claramente el valor añadido por la aplicación o producto.
\end{itemize}

\subsection{Esqueleto}
\begin{itemize}
    \item Diseño de información (software / web)
    \item Diseño de la interfaz (software)
    \item Diseño de la navegación (web)
\end{itemize}
\subsection{Superficie}
\begin{itemize}
    \item Se les debe dar a las personas lo que necesitan o quieren.
    \item Se les debe proporcionar cuando y dónde ellos lo quieran.
    \item Se debe mostrar en un formato visual que asegure que puedan (y quieran) acceder a todo.
    \item Organización del contenido
    \item Diagrama de la pantalla
    \item Uso de audio y video
    \item Como usan las personas las manos y dedos para moverse entre la información.
\end{itemize}

\begin{itemize}
    \item ux y ui es dificil poder hacer bien ambos roles.
\end{itemize}

\subsection{El diseño de interacción define la estructura y el comportamiento de los sistemas.}
\begin{itemize}
    \item Crea relaciones significativas entre las personas y las cosas que usan.
    \item Comunica eficientemente la interactividad y la funcionalidad.
    \item Revela flujos de trabajo simples y complejos.
    \item Informa a los usuarios sobre los cambios de estado.
\end{itemize}

\subsection{Principios fundamentales de IXD}
\begin{itemize}
    \item Si los usuarios pueden contestar.
    \item ¿Donde estoy?
    \item ¿Cómo llegué aquí?
    \item Informa a los usuarios sobre los cambios de estado.
    \item Vista previa de la estructura para proveer el contexto. 
    \item Muestra lo que se puede hacer con la app.
    \item Las etiquetas, instrucciones, iconos e imágenkes generan expectativas sobre:
    \item ¿qué hacer?
    \item ¡qué pasará!
    \item ¿a dónde irá el usuario?
    \item ¿Cómo responderá el sistema?
    \item Ubicación: dónde estoy
    \item Estatus - ¿qué está pasando? / ¿qué sigue ocurriendo?
    \item estado futuro - ¿Qué pasará?
    \item resultados - esto es lo que ha ocurrido.
    \item cada acción debe producir una reacción visible, ser entendible e inmediata.
    \item Reconoce interacciones para hacerle saber a las personas que han sido escuchadas (o sentidas, o vistas).
    \item Prevención de errores es la mejor manera de manejarlos.
    \item Los mensajes de error deben:
    \item Describir que paso 
    \item Porque ha pasado
    \item Sugerir soluciones
    \item wireframes, mockups prototipos
    \item prototipo es un experimento
    \item un prototipo es versionable, puede tener 100 versiones mientras contesta preguntas de la interfaz
    \item Para presentar preguntas vs prototipar para contestar preguntas
    \item Un proyecto informático nunca termina de desarrollarse.
    \item Sistemas de diseño
    \item En diseño web, se crearon los sistemas de diseño donde defines de manera predeterminada como va a ser toda la estrcutrau de la aplicacion
    
\end{itemize}

\subsection{Qué es un sistema de diseño}
\begin{itemize}
    \item Esta herramienta permite establecer patrones y contar con elemenos que se pueden, y deben, reutilizar para crear funcionalidades.
    \item La modularidad del sistema es lo que permite crear desde una unidad mínima hasta componentes más complejos.
    \item Establece reglas que nos ayudan a trabajar en equipo de forma alineada a través de principios.
    \item Además, el sistema de diseño refleja el punto de unión entre el equipo de diseño y desarrollo.
    \item implementa un lenguaje claro y consistente a partir del cual crear y evolucionar productos.
\end{itemize}

\subsection{Qué valor aporta}
\begin{itemize}
    \item Garantiza la consistencia de nuestros productos.
    \item Es positivo en la experiencia de usuario.
    \item Acorta los tiempos de ideación
\end{itemize}

\subsection{Sistemas de diseño}
\begin{itemize}
    \item Prioriza las tareas principales
    \item Establezca un objetivo principal para sus usuarios en cada 
\end{itemize}

\subsection{Sistemas de diseño}
\begin{itemize}
    \item Es ideal separar cada apartado en una página distinta.
    \item Utiliza nombres reconocibles; usa ícenos si es necesario.
    \item La separación debe ser consistente y coherente.
    \item Tipo de letra, tamaño, separación.
    \item material design es el que utiliza android por ejemplo.
    \item accesibilidad web. 
\end{itemize}

\subsection{Wireframes, mockups y prototipos}
\begin{itemize}
    \item wireframes
    \item Es el esqueleto de una pantalla, muestra la prioridad y organización de elementos en pantalla.
    \item Detalla como los usuarios se moverán en las diversas partes de la app.
\end{itemize}


\end{document}
